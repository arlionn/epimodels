\documentclass{article}
\usepackage {tikz}
\usetikzlibrary {positioning}
\setlength{\parskip}{\baselineskip}%
\definecolor {processblue}{cmyk}{0.96,0,0,0}
   \title{EPIMODELS}

\author{
\textit{Sergiy Radyakin}, Development Economics Data Group, The World Bank \\
\textit{Paolo Verme}, Fragility, Conflict and Violence, The World Bank }

\begin{document}
\maketitle
\section{SEIR model}
\begin {center}
	\begin {tikzpicture}[-latex ,auto ,node distance =5 cm and 4cm ,on grid , semithick ,
	state/.style ={ circle ,top color =white , bottom color = processblue!20 ,
	draw,processblue , text=blue , minimum width =1 cm}]
		\node[state] (S){$S$};
		\node[state] (E) [right=of S] {$E$};
		\node[state] (I) [right =of E] {$I$};
		\node[state] (R) [right =of I] {$R$};
		\path (S) edge [ left =25] node[above]  {$\beta$} (E);
		\path (E) edge [ left =25] node[above]  {$\sigma$} (I);
		\path (I) edge [ left =25] node[above]  {$\gamma$} (R);
		\path (S) edge [bend left =35] node[above]  {$\nu$} (R);
		\path (E) edge [bend left =25] node[above]  {$\mu$} (S);
		\path (I) edge [bend left =35] node[above]  {$\mu$} (S);
		\path (R) edge [bend left =45] node[above]  {$\mu$} (S);
	\end{tikzpicture}
\end{center}
Consider the model defined by the following system of ordinary differential equations: \par
   $ \frac{dS}{dt}=\mu(N-S)-\beta\frac{SI}{N}-\nu S$ , $S(t_0)=S_0$ \par
   $ \frac{dE}{dt}=\beta\frac{SI}{N}-(\mu+\sigma)E$ , $E(t_0)=E_0$ \par
   $ \frac{dI}{dt}=\sigma E  -(\mu + \gamma)I $ , $I(t_0)=I_0$ \par
   $ \frac{dR}{dt}= \gamma I - \mu R + \nu S $ , $R(t_0)=R_0$ \par

The parameter $\beta$ characterizes the speed of contagion, at which susceptible individuals become exposed. The parameter $\sigma$ is the constant rate at which exposed individuals become infected. The parameter $\gamma$ is a constant rate at which infected individuals recover. The parameter $\mu$ is the constant natural mortality rate unrelated to the disease being modelled. The model is formulated under the constant population assumption ($S+E+I+R=N=Population$), so that the natural mortality ($\mu$) is counterbalanced by the equivalent fertility, refreshing the susceptible population. \par

The vaccination rate ($\nu$) is transferring individuals from susceptible state to recovered (assuming resistance to the disease).\par

We can simplify the model by assuming away these two effects (letting both $\mu$ and $\nu$ equal to zero). Under this assumption the model becomes:
\vskip10pt
   $ \frac{dS}{dt}=-\beta\frac{SI}{N}$ , $S(t_0)=S_0$ \par
   $ \frac{dE}{dt}=\beta\frac{SI}{N}-\sigma E$ , $E(t_0)=E_0$ \par
   $ \frac{dI}{dt}=\sigma E  -\gamma I $ , $I(t_0)=I_0$ \par
   $ \frac{dR}{dt}= \gamma I $ , $R(t_0)=R_0$ \par

Note that this effectively becomes a SIR model if we further assume parameter $\sigma$ to be equal to zero and combine the exposed and infected states together:
\vskip10pt
   $ \frac{dS}{dt}=-\beta\frac{SI}{N}$ , $S(t_0)=S_0$ \par
   $ \frac{dI}{dt}=\beta\frac{SI}{N}  -\gamma I $ , $I(t_0)=I_0$ \par
   $ \frac{dR}{dt}= \gamma I $ , $R(t_0)=R_0$ \par


\section{SIR model}

\begin {center}
	\begin {tikzpicture}[-latex ,auto ,node distance =3 cm and 3cm ,on grid , semithick ,
	state/.style ={ circle ,top color =white , bottom color = processblue!20 ,
	draw,processblue , text=blue , minimum width =1 cm}]
		\node[state] (S){$S$};
		\node[state] (I) [right =of S] {$I$};
		\node[state] (R) [right =of I] {$R$};
		\path (S) edge [ left =25] node[above]  {$\beta$} (I);
		\path (I) edge [ left =25] node[above]  {$\gamma$} (R);
	\end{tikzpicture}
\end{center}
The SIR model is also formulated under the constant population assumption $Population=N=S+I+R$ and is characterized by the following system of differential equations:

   $ \frac{dS}{dt}=-\beta\frac{SI}{N}$ , $S(t_0)=S_0$ \par
   $ \frac{dI}{dt}=\beta\frac{SI}{N}  -\gamma I $ , $I(t_0)=I_0$ \par
   $ \frac{dR}{dt}= \gamma I $ , $R(t_0)=R_0$ \par


\end{document}